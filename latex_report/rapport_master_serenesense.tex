\documentclass[12pt,oneside,a4paper]{memoir}

% ---------- Encodage et langue ----------
\usepackage[T1]{fontenc}
\usepackage[utf8]{inputenc}
\usepackage[french]{babel}
\usepackage{csquotes}

% ---------- Mise en page ----------
\usepackage{geometry}
\geometry{margin=2.8cm}
\usepackage{microtype}
\usepackage{setspace}
\setstretch{1.15}

% ---------- Maths, tableaux, graphiques ----------
\usepackage{amsmath,amssymb}
\numberwithin{equation}{chapter}
\usepackage{graphicx}
\usepackage{float}
\usepackage{subcaption}
\usepackage{booktabs}
\usepackage{multirow}
\usepackage{longtable}
\usepackage{array}

% ---------- Code, algorithmes ----------
\usepackage{listings}
\usepackage[ruled,vlined,french,onelanguage]{algorithm2e}

% ---------- Références et hyperliens ----------
\usepackage{xcolor}
\definecolor{sereneBlue}{HTML}{0B3C5D}
\definecolor{sereneGold}{HTML}{C69214}
\usepackage{hyperref}
\hypersetup{
  colorlinks=true,
  linkcolor=sereneBlue,
  citecolor=sereneBlue,
  urlcolor=sereneBlue,
  pdfauthor={Ben Ammar Sirine},
  pdftitle={Mémoire Mastère - Détection Sonore Véhicules NOMAD}
}
\usepackage[noabbrev,nameinlink]{cleveref}
\crefname{chapter}{chapitre}{chapitres}
\crefname{section}{section}{sections}
\crefname{figure}{figure}{figures}
\crefname{table}{table}{tables}
\crefname{equation}{équation}{équations}

% ---------- Bibliographie ----------
\usepackage[backend=biber,style=numeric,sorting=none]{biblatex}
\addbibresource{references.bib}

% ---------- Identité du document ----------
\newcommand{\projecttitle}{Détection Sonore Véhicules NOMAD}
\newcommand{\studentname}{Ben Ammar Sirine}

% ---------- Styles de pages ----------
\makepagestyle{serenesense}
\makeevenhead{serenesense}{\scshape\chaptername~\thechapter}{\scshape\projecttitle}{\scshape\nouppercase{\leftmark}}
\makeoddhead{serenesense}{\scshape\chaptername~\thechapter}{\scshape\projecttitle}{\scshape\nouppercase{\leftmark}}
\makeevenfoot{serenesense}{\scshape\projecttitle}{}{\scshape\studentname}
\makeoddfoot{serenesense}{\scshape\projecttitle}{}{\scshape\studentname}
\pagestyle{serenesense}
\aliaspagestyle{chapter}{serenesense}

% ---------- Styles de chapitres ----------
\chapterstyle{hangnum}
\setsecnumdepth{subsection}
\maxtocdepth{subsection}

% ---------- Listes et codes ----------
\lstset{
  basicstyle=\ttfamily\small,
  keywordstyle=\color{sereneBlue}\bfseries,
  commentstyle=\itshape\color{gray},
  stringstyle=\color{sereneGold},
  frame=single,
  breaklines=true,
  tabsize=2,
  captionpos=b
}

% ---------- Macros utilitaires ----------
\newcommand{\placeholderfigure}[3][]{%
  \begin{figure}[H]
    \centering
    \fbox{%
      \begin{minipage}[c][0.26\textheight][c]{0.9\textwidth}
        \centering
        {\Large Image à insérer : #2}\\[0.5em]
        {\small #3}
      \end{minipage}}
    \caption{#3}
    \label{fig:#1}
  \end{figure}
}

\newcommand{\placetable}[2]{%
  \begin{table}[H]
    \centering
    \begin{tabular}{@{}p{0.35\textwidth}p{0.55\textwidth}@{}}
      \toprule
      Élément & Contenu à insérer \\ \midrule
      #1 \\ \bottomrule
    \end{tabular}
    \caption{#2}
  \end{table}
}

\begin{document}

% ========== Page de titre ==========
\begin{titlingpage}
  \centering
  \vspace*{2cm}
  {\Large Mémoire de Stage de Fin d'Études\\[0.6em]
  Pour l'obtention du Mastère Professionnel en Ingénierie Avancée des Systèmes Robotisés et Intelligence Artificielle}\\[1.4cm]
  {\Huge\bfseries \projecttitle\\[0.8em]
  \normalsize Détection et classification acoustique en milieu militaire}\\[1.1cm]
  \placeholderfigure[fig:logo]{Logo Avionav}{Logo de l'entreprise d'accueil (remplacer par le fichier réel)}\\[0.9cm]
  {\large Présenté par : \textbf{\studentname}}\\[0.4cm]
  {\large Organisme d'accueil : \textbf{Avionav}}\\[0.2cm]
  {\large Encadrants : \textbf{[Nom académique]} \& \textbf{[Nom entreprise]}}\\[1.2cm]
  {\large Année universitaire : 2024--2025}\\
  \vfill
  {\large \today}
\end{titlingpage}

% ========== Préface et sommaire ==========
\frontmatter

\chapter*{Remerciements}
\noindent A la fin de ce travail, je souhaite exprimer ma gratitude à toutes les personnes qui ont, de près ou de loin, contribué au succès de ce projet.\\
\noindent Je tiens, avant tout, à exprimer ma profonde reconnaissance à Monsieur le Professeur Hassene Seddik, mon encadrant pédagogique à l’ENSIT, pour l’attention et l’intérêt qu’il a portés à ce projet. Son suivi constant, ses orientations pertinentes et la qualité de son encadrement ont constitué pour moi une véritable source d’apprentissage et de motivation tout au long de cette expérience.\\
\noindent J’adresse également mes sincères remerciements à Monsieur Foued El Kamel, mon encadrant professionnel au sein de l’entreprise Avionav, pour son accompagnement, sa disponibilité et ses conseils techniques avisés. Son expertise et son sens du professionnalisme ont été essentiels à la réussite de ce travail.\\
\noindent Je remercie également l’ensemble des membres du jury pour l’honneur qu’ils me font en acceptant d’évaluer ce travail. Leurs remarques et suggestions constitueront, sans aucun doute, une contribution précieuse à l’amélioration et à l’approfondissement de mes compétences.\\
\noindent Je tiens aussi à exprimer ma gratitude envers toute l’équipe de l’entreprise Avionav pour leur accueil chaleureux, leur collaboration et leur esprit d’équipe. Leur soutien et leur partage d’expérience ont rendu cette période de stage à la fois enrichissante et formatrice.\\
\noindent Enfin, j’adresse mes plus vifs remerciements à mes professeurs de l’ENSIT, ainsi qu’à ma famille et mes amis, pour leur soutien moral, leurs encouragements constants et leur présence bienveillante tout au long de mon parcours académique.

\chapter*{Dédicaces}
\noindent À ma chère mère,\\
pour son amour inépuisable, sa douceur et ses prières constantes qui m’ont toujours accompagnée dans les moments de doute et d’effort.\\[0.4em]
À mon père,\\
pour sa sagesse, son soutien indéfectible et les valeurs de persévérance et d’honnêteté qu’il m’a transmises.\\[0.4em]
À mon époux,\\
pour sa compréhension, son encouragement permanent et sa présence bienveillante tout au long de mon parcours universitaire.\\[0.4em]
À mon enfant bien-aimé,\\
véritable source de bonheur et de motivation, qui m’inspire chaque jour à donner le meilleur de moi-même.\\[0.4em]
À ma belle-mère,\\
à qui je dois une profonde reconnaissance pour son aide précieuse et son soutien sans faille, grâce auxquels j’ai pu poursuivre mes études de master durant ces deux années.\\[0.8em]
Ce travail est le reflet de votre amour, de votre patience et de votre confiance. Je vous le dédie avec tout mon cœur.

\chapter*{Résumé}
Ce projet de fin d’études, réalisé dans le cadre du Mastère Professionnel à l’ENSIT et à l’UVT en collaboration avec l’entreprise Avionav, porte sur le développement d’un système de détection et de classification sonore destiné au rover NOMAD, conçu pour des missions de surveillance de zones frontalières et pour des usages à finalité militaire.

L’étude vise à reconnaître non seulement la présence et le type de véhicules à partir de leurs signatures acoustiques, mais aussi un ensemble d’événements sonores pertinents en contexte opérationnel : mouvements de troupes, communications vocales, véhicules logistiques, bruit de fond ambiant et activités aériennes (hélicoptères, avions de chasse). Pour atteindre ces objectifs, la base de données \emph{Military Audio Detection} (MAD) a été préparée et structurée (7\,466 enregistrements répartis en 7 classes) puis intégrée dans un pipeline complet d’entraînement et d’évaluation.

Le travail réalisé s’articule autour de trois axes principaux. Premièrement, la mise en place de \textbf{baselines MFCC} : extraction de caractéristiques audio (MFCC, coefficients delta et delta-delta), implémentation et entraînement de réseaux de neurones convolutionnels (CNN) et convolutionnels récurrents (CRNN). Ces modèles atteignent respectivement 66{,}88~\% et 73{,}21~\% d’exactitude sur l’ensemble de validation MAD, établissant une première référence sur le jeu de données. Deuxièmement, l’introduction d’un \textbf{modèle transformeur AudioMAE} opérant sur des spectrogrammes de Mel de taille $128\times128$ sur 10~secondes d’audio : ce modèle atteint 82{,}15~\% d’exactitude de validation, avec une généralisation particulièrement favorable (validation meilleure que l’entraînement), et constitue la contribution principale en termes de performance.

Troisièmement, le projet aboutit à la création d’une \textbf{chaîne logicielle et de déploiement complète} sous forme du dépôt SereneSense : préparation automatisée des données, scripts d’entraînement et d’évaluation, génération de rapports, export du modèle AudioMAE au format ONNX, quantification en entiers 8 bits (INT8) et mise en place d’un pipeline de détection temps réel sur Raspberry~Pi~5. Les mesures effectuées sur poste de développement montrent une latence totale d’environ 46~ms pour le modèle ONNX FP32, et les projections sur Raspberry~Pi~5 indiquent des latences attendues de 260 à 340~ms avec le modèle INT8, pour une consommation mémoire d’environ 800~Mo, ce qui respecte confortablement la contrainte de 500~ms par fenêtre audio.

Les travaux menés démontrent ainsi qu’il est possible de combiner des architectures modernes de type transformeur avec des contraintes d’embarqué, et offrent un socle technique solide pour l’intégration ultérieure de ce module de perception acoustique dans le rover NOMAD. Les perspectives incluent l’enrichissement de la base MAD par de nouveaux scénarios, l’exploration de modèles audio pré-entraînés plus avancés et l’intégration multimodale avec les autres capteurs du robot.

\textbf{Mots clés :} classification sonore, Military Audio Detection, réseau de neurones convolutionnel, réseau de neurones convolutionnel récurrent, AudioMAE, transformeur, déploiement embarqué, Raspberry~Pi~5, rover NOMAD, Python.

\chapter*{Abstract}
This master thesis, carried out within the professional program at ENSIT and UVT in collaboration with Avionav, focuses on the development of a sound detection and classification system for the NOMAD rover dedicated to border surveillance and military-oriented missions.

The goal is to recognize not only the presence and type of vehicles from their acoustic signatures, but also a set of operationally relevant sound events such as radio communications, gunshots, footsteps, bombardments and aerial activity. To achieve this, a representative audio database is built from multiple sources and environments, then processed to feed the training and evaluation of deep learning models.

The work includes the preparation of the Military Audio Detection (MAD) dataset, the extraction of audio features (MFCC, delta and delta-delta coefficients), and the design and training of convolutional neural network and convolutional recurrent neural network models. Their performance is evaluated using overall classification rate, confusion matrices and a confidence rate based on a minimum probability threshold set to 70\%.

Experiments show that CNN and CRNN architectures can effectively discriminate several sound classes in noisy environments, each model offering specific advantages in terms of accuracy and temporal information capture. These results represent a key step towards an intelligent acoustic surveillance module embedded on a mobile rover. The study also opens perspectives for enriching the dataset, improving model performance and deploying the system on resource-constrained embedded platforms.

\textbf{Keywords:} sound classification, Military Audio Detection, convolutional neural network, convolutional recurrent neural network, NOMAD rover, Python, Google Colab.

\tableofcontents*
\listoffigures
\listoftables
\listofalgorithms

% ========== Corps ==========
\mainmatter

\chapter{Introduction générale et état de l'art}

\section{Introduction générale}
L’évolution rapide des technologies d’intelligence artificielle et de traitement du signal a profondément transformé les domaines de la surveillance, de la robotique et de la sécurité. Les systèmes autonomes capables d’analyser leur environnement et de prendre des décisions en temps réel constituent aujourd’hui un enjeu majeur dans les applications civiles et militaires.

Dans ce contexte, le présent projet s’inscrit dans le cadre du développement du rover \textbf{NOMAD}, un véhicule terrestre autonome conçu par l’entreprise Avionav. Contrairement aux approches classiques basées sur la vision artificielle, ce projet vise à doter ce rover d’un module intelligent de détection et de classification des sons, afin de lui permettre d’identifier et d’interpréter en temps réel la présence de véhicules militaires, de mouvements humains suspects ou de toute activité potentiellement dangereuse à proximité de zones sensibles. Une telle capacité renforce l’autonomie du système et ouvre la voie à des applications variées, notamment dans les missions de surveillance des zones frontalières et la détection d’activités anormales à caractère militaire, même dans des environnements où la visibilité est limitée (obscurité, fumée, obstacles).

Le projet repose sur l’analyse de signaux audio et l’exploitation d’algorithmes d’apprentissage profond pour reconnaître diverses classes sonores telles que les bruits de moteurs, les communications radio, les coups de feu, les pas, les explosions ou les activités aériennes. Une base de données audio dédiée est élaborée et traitée afin de permettre l’entraînement, la validation et l’évaluation des modèles de classification.

Ce travail soulève plusieurs défis techniques, notamment la gestion du bruit ambiant, la diversité des signaux enregistrés et la nécessité de concevoir un modèle performant, généralisable et embarquable. Ces contraintes ont guidé les choix méthodologiques à chaque étape du projet, de la préparation des données à la sélection des architectures de réseaux de neurones.

\section{Présentation de l'organisme d'accueil}
Le projet de fin d’études a été réalisé au sein de la société tunisienne \textbf{Avionav}, spécialisée dans la fabrication d’avions légers. Cette entreprise se distingue à la fois par son expertise technique et par son positionnement unique sur le marché national.

\subsection{Présentation de la société}
Avionav est une société à responsabilité limitée opérant dans l’industrie aéronautique. Elle conçoit et fabrique des avions légers destinés à différents usages civils et professionnels. Le tableau~\ref{tab:identite-avionav} synthétise les principales informations relatives à l’entreprise.

\begin{table}[H]
  \centering
  \begin{tabular}{@{}ll@{}}
    \toprule
    Élément & Valeur \\
    \midrule
    Nom de l'entreprise & Avionav \\
    Domaine d'activité & Industrie aéronautique (avions légers) \\
    Statut juridique & Société à responsabilité limitée (SARL) \\
    Produits & Avions légers en composite et aluminium \\
    Site internet & \url{https://www.avionav.net} \\
    Responsable & Foued El Kamel \\
    \bottomrule
  \end{tabular}
  \caption{Identité de la société Avionav.}
  \label{tab:identite-avionav}
\end{table}

\placeholderfigure[fig:logo-avionav]{Logo Avionav}{Logo officiel de la société Avionav}

\subsection{Domaine d’activité}
Au cours de son développement dans le secteur de la construction aéronautique, Avionav s’est imposée sur le marché grâce à son savoir-faire technologique et à son positionnement en tant que seul acteur tunisien spécialisé dans la fabrication d’avions légers. L’atelier de production est particulièrement orienté vers :
\begin{itemize}
  \item les matériaux composites à haute performance,
  \item la tôlerie fine et industrielle,
  \item l’usinage de précision,
  \item la peinture et l’assemblage de structures aéronautiques.
\end{itemize}

\subsection{Services et produits}
Avionav propose principalement deux modèles d’avions distincts, offrant une gamme de produits adaptée à différents besoins :
\begin{itemize}
  \item un modèle en composite de fibre de carbone, baptisé \emph{Rally},
  \item un modèle en aluminium, baptisé \emph{Storm}.
\end{itemize}

\placeholderfigure[fig:avion-rally]{Avion Rally}{Avion Rally de la société Avionav}
\placeholderfigure[fig:avion-storm]{Avion Storm}{Avion Storm de la société Avionav}

\subsection{Historique de la société}
L’évolution d’Avionav peut être résumée par les jalons suivants :
\begin{itemize}
  \item 2007 : fondation de l’entreprise Storm Aircraft par un groupe d’entrepreneurs italiens ;
  \item 2011 : création de la start-up Oxygène Aeronautics par les frères El Kamel ;
  \item 2014 : acquisition complète de Storm Aircraft et changement de nom en Avionav ;
  \item 2015 : exportation vers l’Italie d’avions légers en aluminium et en composite ;
  \item 2016 : participation à la fabrication d’un aéronef amphibie à quatre places en partenariat avec Evada Aircraft ;
  \item 2017 : exportation de plusieurs dizaines d’avions vers l’Europe, l’Asie et l’Afrique ;
  \item 2020 : lancement d’un programme d’expansion des ateliers de fabrication et de programmation, avec recrutement de nouvelles compétences.
\end{itemize}

\section{Contexte général}
Au cours des dernières années, les systèmes de surveillance et de reconnaissance autonomes ont connu un développement rapide, porté par les progrès de la robotique, de l’intelligence artificielle et des capteurs embarqués. Ces systèmes sont déployés dans de nombreux domaines : exploration scientifique, surveillance militaire, sécurité civile ou encore suivi environnemental.

Les plateformes robotiques mobiles, telles que les rovers et les systèmes de type Nomad, sont capables d’évoluer sur des terrains complexes, isolés ou difficilement accessibles. Elles intègrent traditionnellement des capteurs visuels, infrarouges ou radar. La perception acoustique, longtemps sous-exploitée, émerge aujourd’hui comme un moyen complémentaire particulièrement intéressant, notamment lorsque la visibilité est limitée ou dégradée.

Dans ce cadre, doter un rover tel que NOMAD d’une capacité de détection et de classification acoustique constitue un atout majeur pour la surveillance et la reconnaissance de situations à risque, indépendamment des conditions météo ou de luminosité.

\section{Présentation du projet}
Le projet intitulé \og Détection sonore pour le véhicule NOMAD \fg{} consiste à mettre en place un rover mobile et autonome intégrant un système intelligent de détection et de classification sonore basé sur des techniques d’intelligence artificielle. Ce véhicule terrestre doit être capable d’identifier et d’interpréter en temps réel des événements sonores d’intérêt, tels que :
\begin{itemize}
  \item la présence de véhicules militaires,
  \item des mouvements humains suspects,
  \item des tirs d’armes à feu ou des explosions,
  \item des activités aériennes (hélicoptères, avions de chasse),
  \item des communications radio ou vocales.
\end{itemize}

Le prototype envisagé comprend :
\begin{itemize}
  \item une plateforme mobile de type Nomad (robot tout-terrain autonome) ;
  \item des capteurs acoustiques (microphones omnidirectionnels) ;
  \item un modèle d’apprentissage profond de type réseau de neurones convolutionnel ou réseau de neurones convolutionnel récurrent pour la détection et la classification ;
  \item un système embarqué à ressources limitées (par exemple Raspberry~Pi~5 ou Jetson Xavier).
\end{itemize}

\section{Problématique}
Malgré les progrès significatifs en robotique et en intelligence artificielle, la plupart des systèmes de surveillance actuels reposent principalement sur la vision artificielle et exploitent peu la perception acoustique. La question centrale est donc de savoir comment mettre en place un système acoustique autonome et mobile répondant aux contraintes du terrain.

Plusieurs défis scientifiques et techniques se posent :
\begin{itemize}
  \item comment choisir ou constituer une base de données représentative du contexte de surveillance (sons militaires, bruits de moteurs, tirs, explosions) pour entraîner efficacement les modèles ;
  \item quelles architectures de modèles d’apprentissage profond adopter pour garantir de bonnes performances de classification malgré la variabilité des conditions acoustiques ;
  \item comment optimiser les modèles pour un déploiement embarqué, en respectant les contraintes de mémoire, de calcul et de consommation énergétique.
\end{itemize}

\section{Objectifs du projet}
L’objectif global du projet est la mise en place d’un prototype de robot NOMAD de surveillance acoustique intégrant un module intelligent de détection et de classification sonore basé sur l’apprentissage profond.

Pour atteindre cet objectif, plusieurs objectifs spécifiques sont définis :
\begin{itemize}
  \item sélectionner ou construire une base de données sonore représentative des sons cibles (véhicules, hélicoptères, bombardements, tirs, etc.) ;
  \item concevoir et entraîner deux architectures d’apprentissage profond, l’une de type réseau de neurones convolutionnel et l’autre de type réseau de neurones convolutionnel récurrent ;
  \item évaluer les performances des modèles à l’aide d’indicateurs quantitatifs (taux de bonne classification, matrices de confusion, courbes d’apprentissage) ;
  \item choisir le modèle le plus adapté à une implémentation sur carte embarquée (Raspberry~Pi~5 ou Jetson Xavier) ;
  \item préparer un prototype fonctionnel capable de détecter et de classifier des sons en temps réel pour un rover de surveillance.
\end{itemize}

\section{Plan de travail}
Le projet est organisé de manière progressive, selon les principales tâches suivantes :
\begin{itemize}
  \item recherche bibliographique et technologique sur la robotique mobile et la classification sonore ;
  \item collecte, préparation et nettoyage des données audio ;
  \item conception des architectures de réseaux de neurones ;
  \item entraînement et validation des modèles ;
  \item évaluation et comparaison des performances ;
  \item préparation à l’implémentation embarquée ;
  \item analyse des résultats et rédaction du mémoire.
\end{itemize}

\section{Robots mobiles et systèmes de type Nomad}
Les robots mobiles autonomes, tels que les rovers et les robots de type Nomad, constituent le socle de nombreux systèmes d’exploration et de surveillance. Ils sont conçus pour se déplacer dans des environnements hostiles ou difficiles d’accès, tout en emportant des capteurs et des systèmes de communication.

\subsection{Robots de type rover}
Les rovers sont des véhicules robotiques terrestres conçus pour se déplacer sur des terrains souvent accidentés ou non structurés afin d’explorer, d’inspecter ou d’effectuer des tâches dans des environnements dangereux ou inaccessibles. Ils sont généralement équipés de roues ou de chenilles, ainsi que de capteurs embarqués (caméras, capteurs inertiels, systèmes de navigation).

Parmi leurs principaux domaines d’application, on peut citer :
\begin{itemize}
  \item l’exploration spatiale (par exemple Sojourner, Spirit, Perseverance) ;
  \item les missions de sauvetage en zones sinistrées (PackBot) ;
  \item la recherche environnementale et agricole (TerraSentia) ;
  \item les applications militaires et sécuritaires (TALON, Ripsaw, Crusher).
\end{itemize}

\placeholderfigure[fig:rover-exemples]{Exemples de rovers}{Illustrations de rovers d’exploration, de sauvetage et militaires}

\subsection{Robots de type Nomad}
Les robots de type Nomad sont conçus pour se déplacer de manière autonome sur de grandes distances dans des environnements complexes, imprévisibles et hostiles. Ils combinent :
\begin{itemize}
  \item des capteurs variés (caméras, LIDAR, GPS, microphones, infrarouge) ;
  \item une intelligence embarquée pour la planification, la navigation et la prise de décision ;
  \item des systèmes de locomotion robustes (roues, chenilles ou pattes) adaptés aux terrains difficiles ;
  \item des moyens de communication longue portée.
\end{itemize}

Ces plateformes sont utilisées pour l’exploration scientifique terrestre (déserts, pôles, volcans), la surveillance environnementale (par exemple Wave Glider pour l’océanographie) et la défense (par exemple GuardBot pour la surveillance de zones sensibles). Le projet NOMAD d’Avionav s’inscrit dans cette lignée, en mettant l’accent sur la perception acoustique.

\section{Systèmes de surveillance acoustique existants}
Contrairement aux systèmes robotiques précédents, les solutions de surveillance acoustique existantes se concentrent sur la détection et la classification de sons, indépendamment de la mobilité de la plateforme. Elles sont utilisées dans des domaines variés : sécurité, défense, surveillance environnementale ou industrielle.

\subsection{Systèmes de détection acoustique de tirs}
Les systèmes de détection de tirs sont capables de détecter le son d’un tir et de localiser son origine. Ils s’appuient sur des réseaux de microphones et des techniques de triangulation et de corrélation temporelle. Parmi les exemples connus, on peut citer :
\begin{itemize}
  \item \emph{Boomerang}, développé pour les véhicules militaires américains, capable de détecter en temps réel la direction et la distance d’un tir, même en environnement bruyant ;
  \item \emph{ShotSpotter}, réseau de capteurs installé dans plusieurs villes pour repérer et localiser les coups de feu en milieu urbain.
\end{itemize}

\placeholderfigure[fig:boomerang]{Système Boomerang}{Exemple de système de détection acoustique de tirs}
\placeholderfigure[fig:shotspotter]{Système ShotSpotter}{Exemple de réseau urbain de détection de tirs}

\subsection{Systèmes de détection sous-marins}
Les systèmes acoustiques sous-marins (sonars) sont utilisés pour surveiller et détecter des objets ou des événements dans l’environnement marin. Ils reposent sur des réseaux d’hydrophones capables de capter les ondes acoustiques émises par des sous-marins, des navires ou d’autres sources. On distingue :
\begin{itemize}
  \item les sonars passifs, qui se contentent d’écouter les bruits produits par les cibles ;
  \item les sonars actifs, qui émettent un signal acoustique et analysent l’écho réfléchi.
\end{itemize}

\subsection{Systèmes de surveillance environnementale et industrielle}
Les systèmes acoustiques sont également employés pour surveiller des zones naturelles ou industrielles sensibles. Ils permettent, par exemple, de détecter des mouvements sismiques, des fuites de gaz ou des pollutions sonores. Des dispositifs comme \emph{AudioMoth} sont utilisés en recherche environnementale pour enregistrer des sons d’animaux ou surveiller des forêts tropicales.

\placeholderfigure[fig:audiomoth]{Dispositif AudioMoth}{Exemple de capteur acoustique pour la surveillance environnementale}

\subsection{Systèmes mobiles de surveillance acoustique}
Enfin, certaines plateformes robotiques récentes embarquent des capteurs acoustiques afin de compléter la perception visuelle ou infrarouge. Des drones intégrant des modules de détection acoustique peuvent ainsi localiser des bruits de moteurs ou d’autres sources sonores depuis une plateforme aérienne.

\section{Analyse critique de l’existant}
L’analyse des systèmes existants montre que la détection acoustique est déjà utilisée dans de nombreux domaines (défense, industrie, environnement), mais présente plusieurs limites lorsqu’il s’agit de l’intégrer dans des plateformes mobiles autonomes :
\begin{itemize}
  \item la majorité des systèmes robotiques reposent sur la vision et le GPS, avec une intégration encore limitée de capteurs acoustiques ;
  \item les dispositifs acoustiques sont souvent conçus pour des environnements relativement contrôlés, ce qui complique la reconnaissance en milieu fortement bruité ;
  \item l’autonomie énergétique reste un défi pour les dispositifs de surveillance acoustique en continu ;
  \item l’adaptation à des situations dynamiques (sons changeants, milieux ouverts) est encore perfectible.
\end{itemize}

Ces constats justifient le développement d’un système mobile autonome combinant mobilité, perception acoustique et traitement embarqué, comme proposé dans le projet NOMAD.

\section{Approches de classification sonore}
Les approches de classification sonore peuvent être regroupées en deux grandes familles : les approches classiques fondées sur l’extraction manuelle de caractéristiques, et les approches récentes fondées sur l’apprentissage profond.

\subsection{Approches classiques fondées sur l’apprentissage automatique}
Les approches classiques reposent sur l’extraction manuelle de descripteurs acoustiques (par exemple les coefficients cepstraux en fréquence de Mel, les spectrogrammes ou le taux de passages par zéro), suivie de l’utilisation de classificateurs supervisés comme les machines à vecteurs de support ou les forêts aléatoires.

\subsubsection{Coefficients cepstraux et machine à vecteurs de support}
La combinaison des coefficients cepstraux en fréquence de Mel et des machines à vecteurs de support permet d’obtenir des systèmes de classification sonore fiables et relativement peu coûteux en calcul. Plusieurs travaux récents rapportent des précisions de l’ordre de 75~\% à 90~\% selon les jeux de données considérés.

\subsubsection{Forêt aléatoire}
Les forêts aléatoires sont également largement utilisées pour la classification sonore. Elles sont robustes au bruit et faciles à entraîner, mais ne capturent pas directement la dimension temporelle des sons. Dans la littérature, des performances élevées sont rapportées pour certaines tâches spécifiques (par exemple la classification de sons cardiaques) lorsque les caractéristiques d’entrée sont bien choisies.

\subsection{Approches récentes fondées sur l’apprentissage profond}
Les approches basées sur l’apprentissage profond apprennent directement à partir de représentations temps–fréquence (spectrogrammes ou coefficients cepstraux), ce qui réduit la dépendance à l’ingénierie de caractéristiques manuelle et a permis des gains importants en classification sonore.

\subsubsection{Réseaux de neurones convolutionnels}
Les réseaux de neurones convolutionnels appliqués aux spectrogrammes audio permettent d’extraire automatiquement des motifs locaux pertinents pour la reconnaissance de sons. De nombreuses études montrent qu’ils surpassent les approches classiques sur des tâches de classification multi-classes, notamment en présence de bruit.

\subsubsection{Réseaux récurrents et réseaux convolutionnels récurrents}
Les réseaux de neurones récurrents (et leurs variantes à mémoire longue ou à unités à portes) sont adaptés aux données séquentielles et permettent de modéliser la dynamique temporelle des sons. Les architectures hybrides combinant convolutions et couches récurrentes (réseaux de neurones convolutionnels récurrents) exploitent à la fois l’information fréquentielle et temporelle, et sont particulièrement adaptées aux scènes sonores complexes.

\subsubsection{Modèles pré-entraînés et transformeurs pour l’audio}
Plus récemment, des modèles pré-entraînés sur de très grandes bases de données audio ont été proposés, tels que YamNet ou des architectures de type transformeur dédiées aux spectrogrammes audio \cite{gong2021ast,huang2022audiomae,chen2022beats}. Ces modèles exploitent des mécanismes d’attention et des stratégies d’apprentissage auto-supervisé pour obtenir des représentations riches et transférables.

\section{Comparaison des approches et justification du choix}
La comparaison des différentes approches met en évidence plusieurs compromis entre précision, complexité, capacité à modéliser la dimension temporelle et facilité de déploiement embarqué. Les méthodes classiques sont simples et légères, mais limitées dès que les scènes sonores deviennent complexes. Les architectures profondes récentes, en particulier les réseaux convolutionnels et les réseaux convolutionnels récurrents, offrent de meilleures performances au prix d’une complexité accrue.

Dans le cadre de ce projet, le choix s’est porté sur deux modèles :
\begin{itemize}
  \item un réseau de neurones convolutionnel, pour l’extraction efficace de motifs à partir des spectrogrammes ;
  \item un réseau de neurones convolutionnel récurrent, pour mieux capturer la dimension temporelle des sons militaires.
\end{itemize}

Cette stratégie, déjà explorée dans des travaux antérieurs sur MAD et consolidée dans le cadre du projet SereneSense \cite{benammar2025serenesense}, permet de comparer les performances des deux architectures et de sélectionner celle qui offre le meilleur compromis entre précision, robustesse et faisabilité sur carte embarquée.

\section{Bases de données sonores}

\subsection{Principales bases de données pour la classification sonore}
Pour entraîner et évaluer les modèles de classification sonore, plusieurs bases de données publiques sont utilisées dans la littérature : ESC-50 pour les sons environnementaux \cite{esc50_dataset}, UrbanSound8K pour les sons urbains \cite{urbansound8k_dataset} ou AudioSet pour les scènes sonores variées \cite{gemmeke2017audioset}. Ces jeux de données constituent des références pour le benchmarking de modèles.

Toutefois, ils ne couvrent pas spécifiquement les sons militaires (tirs, explosions, véhicules blindés, hélicoptères) dans des conditions de terrain réalistes.

\subsection{Évaluation et choix de la base de données MAD}
Pour ce projet, la base de données \emph{Military Audio Detection} (MAD) a été retenue \cite{kim2024mad}. Elle contient exclusivement des sons militaires et industriels pertinents pour la surveillance acoustique : bruits de moteurs, véhicules militaires, hélicoptères, avions de chasse, communications et bruits de fond.

Le dataset MAD regroupe plusieurs milliers d’extraits audio répartis en sept classes et enregistrés dans des conditions proches du terrain opérationnel. Il constitue ainsi une base adaptée à l’entraînement et à l’évaluation de modèles ciblant la surveillance acoustique militaire.

\section{Synthèse et solution adoptée}
À l’issue de l’étude de l’existant, la solution adoptée repose sur :
\begin{itemize}
  \item l’utilisation de la base de données MAD, représentative des sons militaires dans des conditions réalistes ;
  \item la conception et l’évaluation de deux architectures d’apprentissage profond (réseau convolutionnel et réseau convolutionnel récurrent) adaptées à la classification de spectrogrammes audio ;
  \item l’intégration future de ces modèles dans un système embarqué sur rover NOMAD, en veillant à respecter les contraintes de latence et de consommation de ressources.
\end{itemize}

\section{Contributions du mémoire}
Au regard de cette synthèse, les contributions concrètes de ce mémoire peuvent être résumées comme suit :
\begin{itemize}
  \item \textbf{Reproduction contrôlée du CNN historique}~: ré-implémentation et validation d’un réseau convolutionnel MFCC sur MAD, en reproduisant les performances de la première étude (exactitude de validation \textasciitilde66{,}88~\%) dans un code structuré et testable.
  \item \textbf{Amélioration par réseau convolutionnel récurrent (CRNN)}~: conception, implémentation et entraînement d’un CRNN MFCC atteignant \textasciitilde73{,}21~\% d’exactitude de validation, démontrant l’apport de la modélisation temporelle explicite pour les scènes sonores militaires.
  \item \textbf{Introduction et évaluation d’un modèle AudioMAE}~: intégration d’une architecture transformeur de type AudioMAE opérant sur des spectrogrammes de Mel 128$\times$128, entraînée sur MAD et atteignant 82{,}15~\% d’exactitude de validation, avec une excellente capacité de généralisation.
  \item \textbf{Mise en place du pipeline SereneSense}~: développement d’un dépôt logiciel complet (préparation de données, scripts d’entraînement/évaluation, configurations YAML, rapports, visualisations) garantissant la reproductibilité des expériences et la traçabilité des résultats.
  \item \textbf{Préparation du déploiement sur Raspberry~Pi~5}~: conception et implémentation d’une chaîne de déploiement (export ONNX, quantification INT8, scripts et documentation) démontrant la faisabilité temps réel du modèle AudioMAE sur une plateforme embarquée à ressources limitées.
\end{itemize}

\section{Structure du mémoire}
Pour répondre à la remarque de fusion des chapitres~1 et~2, ce premier chapitre regroupe l’introduction générale, la présentation du projet et l’état de l’art technologique et scientifique. La suite du mémoire est organisée comme suit :
\begin{description}
  \item[Chapitre~\ref{chap:dataset}] présentation détaillée du jeu de données MAD, de sa structure, de la préparation et du prétraitement des signaux audio ;
  \item[Chapitre~\ref{chap:methodology}] description de la méthodologie et des architectures retenues (réseau convolutionnel, réseau convolutionnel récurrent et modèle transformeur AudioMAE), ainsi que des hyperparamètres, équations utilisées et choix d’optimisation ;
  \item[Chapitre~\ref{chap:results}] présentation des résultats expérimentaux pour les trois modèles (CNN, CRNN, AudioMAE), des métriques (taux de bonne classification, taux de confiance, matrices de confusion) et interprétation détaillée des performances et de la généralisation ;
  \item[Chapitre~\ref{chap:deployment}] discussion de l’implémentation logicielle et du plan de déploiement sur systèmes embarqués (export ONNX, quantification INT8, préparation du déploiement sur carte Raspberry~Pi~5 et pipeline d’inférence temps réel) ;
  \item[Chapitre~\ref{chap:discussion}] analyse critique du travail réalisé, des limites et des perspectives d’amélioration ;
  \item[Chapitre~\ref{chap:conclusion}] conclusion générale et perspectives futures.
\end{description}

\chapter{Jeu de données et prétraitements}\label{chap:dataset}

\section{Présentation du jeu de données Military Audio Detection}
Le jeu de données \textbf{Military Audio Detection (MAD)} a été conçu pour des applications de surveillance acoustique militaire. Il comprend 7\,466 enregistrements audio mono rééchantillonnés à 16~kHz, couvrant des scènes de quelques secondes jusqu’à une dizaine de secondes. Dans le cadre du projet SereneSense, les signaux sont préparés sous forme de segments de 10~secondes pour les expériences principales avec AudioMAE, ce qui permet de capturer davantage de contexte temporel (approche d’un véhicule, phases de régime moteur, etc.).

Les enregistrements sont répartis en sept classes représentatives, alignées sur la définition adoptée dans le projet SereneSense :
\begin{itemize}
  \item \textbf{Hélicoptère} : aéronefs à voilure tournante ;
  \item \textbf{Avion de chasse} : jets militaires à voilure fixe ;
  \item \textbf{Véhicule militaire} : véhicules blindés, transport de troupes, engins chenillés ;
  \item \textbf{Camion} : camions militaires logistiques ;
  \item \textbf{Mouvements à pied} : déplacements d’infanterie, pas humains ;
  \item \textbf{Parole} : conversations, communications vocales en environnement militaire ;
  \item \textbf{Bruit de fond} : ambiance sonore (vent, environnement urbain ou rural sans événement spécifique).
\end{itemize}

Pour les expériences menées dans ce mémoire, le jeu de données est réparti en trois sous-ensembles suivant la configuration adoptée dans le projet SereneSense :
\begin{itemize}
  \item \textbf{Entraînement} : 5\,464 échantillons (73{,}2~\%) ;
  \item \textbf{Validation} : 965 échantillons (12{,}9~\%) ;
  \item \textbf{Test} : 1\,037 échantillons (13{,}9~\%).
\end{itemize}
Ces valeurs correspondent à la préparation finale automatisée réalisée par les scripts \texttt{prepare\_data.py} et \texttt{prepare\_mad\_metadata.py}, qui construisent les fichiers HDF5 pour chaque sous-ensemble et génèrent un fichier de métadonnées (distribution des classes, comptage par split) utilisé ensuite par les pipelines d’entraînement et d’évaluation.

\begin{table}[H]
  \centering
  \begin{tabular}{@{}lccc@{}}
    \toprule
    Classe & Entraînement & Validation & Test \\
    \midrule
    Hélicoptère & -- & -- & -- \\
    Avion de chasse & -- & -- & -- \\
    Véhicules militaires & -- & -- & -- \\
    Camions & -- & -- & -- \\
    Mouvements à pied & -- & -- & -- \\
    Parole & -- & -- & -- \\
    Bruit de fond & -- & -- & -- \\
    \midrule
    Total & 5\,464 & 965 & 1\,037 \\
    \bottomrule
  \end{tabular}
  \caption{Distribution par classe (remplir avec les valeurs exactes).}
  \label{tab:distribution-classes}
\end{table}

\placeholderfigure[fig:distribution-classes]{Distribution des classes}{Histogramme de la distribution par classe (à insérer)}
\placeholderfigure[fig:waveform-exemples]{Exemples de signaux}{Formes d'onde représentatives par classe}

\section{Prétraitements audio}
Les signaux audio bruts sont d’abord normalisés et convertis en représentations temps–fréquence adaptées à l’apprentissage profond. Le pipeline adopté s’inspire des approches classiques en reconnaissance de la parole et comprend les étapes suivantes.

\subsection{Transformée de Fourier à court terme}
Le signal discret $x[n]$ est découpé en trames temporelles de longueur $N$ avec un recouvrement $H$. Pour chaque trame $t$, on applique une fenêtre de Hann $w[n]$ puis la transformée de Fourier à court terme :
\begin{equation}
X(k,t) = \sum_{n=0}^{N-1} x[n + tH]\, w[n]\, e^{-j 2 \pi k n / N},
\end{equation}
où $k$ désigne l’indice fréquentiel.

\subsection{Projection sur l’échelle de Mel}
Le spectre de puissance est ensuite projeté sur un banc de filtres triangulaires suivant l’échelle de Mel afin de se rapprocher de la perception humaine des fréquences. Le spectrogramme de puissance sur l’échelle de Mel $S_{\text{mel}}(m,t)$ est donné par :
\begin{equation}
S_{\text{mel}}(m, t) = \sum_{k = k_{\min}}^{k_{\max}} H_{m}(k)\, \lvert X(k,t)\rvert^{2},
\end{equation}
où $H_{m}(k)$ est la réponse du $m$-ème filtre de Mel.

\subsection{Coefficients cepstraux en fréquence de Mel}
Les énergies passent ensuite en échelle logarithmique, puis une transformée en cosinus discrète est appliquée pour obtenir les coefficients cepstraux en fréquence de Mel (MFCC). Pour un temps $t$ et un indice de coefficient $c$, on a :
\begin{equation}
C(c,t) = \sum_{m=1}^{M} \log\bigl(S_{\text{mel}}(m,t)\bigr)\,
        \cos\left[\frac{\pi c}{M}\left(m-\tfrac{1}{2}\right)\right],
\end{equation}
où $M$ est le nombre de filtres de Mel.

\subsection{Coefficients delta et delta-delta}
Afin de capturer la dynamique temporelle, on calcule la dérivée première (coefficients delta) puis la dérivée seconde (coefficients delta-delta) des MFCC :
\begin{equation}
\Delta C(c,t) =
 \frac{\sum_{n=1}^{K} n\bigl(C(c,t+n) - C(c,t-n)\bigr)}
      {2 \sum_{n=1}^{K} n^{2}},
\end{equation}
\begin{equation}
\Delta^{2} C(c,t) =
 \frac{\sum_{n=1}^{K} n\bigl(\Delta C(c,t+n) - \Delta C(c,t-n)\bigr)}
      {2 \sum_{n=1}^{K} n^{2}},
\end{equation}
où $K$ est l’ordre de la fenêtre temporelle utilisée pour l’approximation.

\subsection{Normalisation et construction du tenseur d’entrée}
Les trois matrices $C(c,t)$, $\Delta C(c,t)$ et $\Delta^{2} C(c,t)$ sont normalisées par standardisation (moyenne nulle et variance unitaire) puis empilées pour former un tenseur tridimensionnel de taille $(F \times T \times 3)$, où $F$ est le nombre de coefficients retenus et $T$ le nombre de trames temporelles. Ce tenseur constitue l’entrée des modèles convolutionnel et convolutionnel récurrent.

\placeholderfigure[fig:mel-exemple]{Représentation temps–fréquence}{Exemple de représentation MFCC (avec delta et delta-delta) pour une classe véhicule}

\section{Augmentations et régularisation des données}
Pour améliorer la robustesse :
\begin{itemize}
  \item un masquage temporel et fréquentiel de type \emph{SpecAugment} appliqué sur les représentations temps–fréquence ;
  \item l’ajout contrôlé de bruit de fond pour simuler des environnements plus bruités ;
  \item une normalisation systématique des caractéristiques pour centrer et réduire les distributions.
\end{itemize}

Ces augmentations visent à réduire le surapprentissage et à améliorer la capacité de généralisation des modèles en présence de conditions acoustiques variées.

\section{Stratégie de séparation des données}
La séparation entraînement/validation/test conserve la distribution des classes et évite les recouvrements de scènes acoustiques similaires entre les ensembles. La présence d’un ensemble de validation dédié (environ 10~\% du total) permet d’estimer la capacité de généralisation des modèles et d’ajuster les hyperparamètres sans biais sur le jeu de test final.

\section{Nettoyage et contrôle qualité}
Les enregistrements bruités ou tronqués sont filtrés en amont. Les fichiers invalides (durée $<9{,}5~\text{s}$ ou absence de signature spectrale) sont exclus. Les méta-données (classe, durée, SNR estimée) sont conservées pour la reproductibilité et la possibilité d'entraînement conditionnel ultérieur.

\chapter{Méthodologie et architectures}\label{chap:methodology}

\section{Approche globale}
La méthodologie adoptée dans ce projet est volontairement \textbf{incrémentale} et s’articule autour de trois niveaux d’architecture, en cohérence avec le dépôt de code SereneSense :
\begin{enumerate}
  \item \textbf{Phase~1~: baselines MFCC (CNN/CRNN).} Concevoir d’abord un modèle de base de type réseau de neurones convolutionnel exploitant des tenseurs MFCC (avec coefficients delta et delta-delta), puis enrichir cette architecture par l’ajout de couches récurrentes bidirectionnelles (CRNN) afin de mieux modéliser la dimension temporelle. Ces modèles constituent les \emph{baselines historiques} reproduisant et améliorant les premiers travaux réalisés sur MAD.
  \item \textbf{Phase~2~: modèle transformeur AudioMAE.} Passer ensuite à une architecture moderne de type transformeur, AudioMAE (\emph{Masked Autoencoders that Listen}), appliquée à des spectrogrammes de Mel en entrée. Ce modèle représente l’état de l’art pour la classification audio et constitue la contribution principale en termes de performance.
  \item \textbf{Phase~3~: industrialisation et déploiement.} Intégrer ces modèles dans une chaîne complète de traitement comprenant préparation des données, scripts d’entraînement, évaluation, export ONNX, quantification INT8 et déploiement sur Raspberry~Pi~5 pour l’inférence temps réel.
\end{enumerate}

Les trois familles de modèles partagent un pipeline de prétraitement cohérent (chapitre~\ref{chap:dataset}) et des outils communs (scripts Python, fichiers de configuration YAML, journaux d’entraînement), ce qui garantit la comparabilité de leurs performances et la reproductibilité des résultats.

\section{Architectures étudiées}
\subsection{Réseau de neurones convolutifs de référence}
Le réseau convolutionnel de référence traite le tenseur d’entrée de taille $(F \times T \times 3)$ issu des coefficients MFCC, delta et delta-delta. L’architecture adopte une structure classique :
\begin{itemize}
  \item trois blocs \emph{convolution–normalisation–fonction d’activation} avec des noyaux de petite taille (par exemple $3\times3$) ;
  \item des couches de sous-échantillonnage (max-pooling) pour réduire progressivement la dimension spatiale et extraire les caractéristiques les plus pertinentes ;
  \item un aplatissement des cartes de caractéristiques suivi d’une ou plusieurs couches entièrement connectées ;
  \item une couche de sortie de dimension égale au nombre de classes (sept dans le cas de MAD), avec une activation \emph{softmax}.
\end{itemize}

Ce modèle compte de l’ordre de quelques centaines de milliers de paramètres (environ 242~k dans la configuration retenue) et constitue la base de comparaison pour la suite. Dans le cadre du projet, il est ré-implémenté de manière modulaire dans le dossier \texttt{src/core/models/legacy}, avec une configuration entièrement décrite dans \texttt{configs/models/legacy\_cnn\_mfcc.yaml}.

\placeholderfigure[fig:cnn-arch]{Architecture du réseau convolutionnel}{Schéma simplifié du réseau convolutionnel (blocs convolution–pooling puis couches denses)}

\subsection{Réseau convolutionnel récurrent}
Le réseau convolutionnel récurrent reprend la partie convolutionnelle précédente pour l’extraction de caractéristiques locales, puis ajoute des couches récurrentes pour modéliser les dépendances temporelles :
\begin{itemize}
  \item les cartes de caractéristiques issues des convolutions sont réorganisées sous forme de séquence temporelle ;
  \item deux couches de réseaux de neurones récurrents bidirectionnels (par exemple BiLSTM) traitent cette séquence afin de capturer les transitions et la dynamique des événements sonores ;
  \item une couche entièrement connectée et une couche de sortie \emph{softmax} produisent les probabilités associées aux sept classes.
\end{itemize}

Ce modèle comporte davantage de paramètres que le réseau purement convolutionnel (environ 1{,}5~M de paramètres dans la configuration retenue), mais permet une meilleure prise en compte de la structure temporelle des sons militaires. Comme pour le CNN, l’implémentation est factorisée dans le code (\texttt{CNNMFCCModel}, \texttt{CRNNMFCCModel}) et entièrement pilotée par un fichier YAML (\texttt{legacy\_crnn\_mfcc.yaml}), ce qui facilite la reproductibilité des expériences.

\placeholderfigure[fig:crnn-arch]{Architecture du réseau convolutionnel récurrent}{Schéma simplifié combinant blocs convolutionnels et couches récurrentes bidirectionnelles}

\begin{table}[H]
  \centering
  \begin{tabular}{@{}lccc@{}}
    \toprule
    Modèle & Paramètres & Entrée & Particularités \\
    \midrule
    Réseau convolutionnel & \textasciitilde242\,K & MFCC + delta + delta-delta & Convolution + sous-échantillonnage \\
    Réseau convolutionnel récurrent & \textasciitilde1{,}5\,M & MFCC + delta + delta-delta & Convolution + BiLSTM bidirectionnel \\
    \bottomrule
  \end{tabular}
  \caption{Spécifications des architectures étudiées.}
  \label{tab:modeles}
\end{table}

\subsection{Modèle AudioMAE : transformeur pour spectrogrammes}
La contribution majeure du projet est l’adoption d’un modèle \textbf{AudioMAE} (\emph{Audio Masked Autoencoder}), inspiré des transformeurs de type Vision Transformer (ViT) et adapté à l’audio. Contrairement aux baselines MFCC, ce modèle opère directement sur des spectrogrammes de Mel bidimensionnels de taille $128\times128$ calculés à partir de segments audio de 10~secondes.

L’architecture se décompose en deux parties :
\begin{itemize}
  \item un \textbf{encodeur} transformeur profond (12 couches, dimension d’embedding 768, 12 têtes d’attention) opérant sur une séquence de patchs $16\times16$ extraits du spectrogramme de Mel ;
  \item un \textbf{décodage} utilisé en phase de pré-entraînement (8 couches, dimension 512, 16 têtes) qui permet de reconstruire les patchs masqués ; pour la classification supervisée sur MAD, la tête de classification est branchée directement sur le jeton de classification (\texttt{[CLS]}) en sortie de l’encodeur.
\end{itemize}

Le modèle complet compte environ 111~millions de paramètres. Il est implémenté dans le module \texttt{src/core/models/audioMAE}, et les hyperparamètres sont définis dans le fichier \texttt{configs/models/audioMAE.yaml}. Le choix de cette architecture se justifie par :
\begin{itemize}
  \item sa capacité à capturer des dépendances de long terme dans les séquences audio (10~secondes) grâce au mécanisme d’attention ;
  \item l’utilisation de représentations apprises (spectrogrammes de Mel + embeddings) mieux adaptées que des descripteurs strictement fixés comme les MFCC ;
  \item la possibilité de réutiliser les mêmes blocs pour un futur pré-entraînement auto-supervisé sur de grandes bases audio (AudioSet, FSD50K).
\end{itemize}

\section{Fonction de coût et métrique principale}
La tâche étant une classification multi-classes, la fonction de coût utilisée est l’entropie croisée catégorielle. Pour une observation donnée, avec un vecteur cible $y$ et un vecteur de probabilités prédites $p$, la perte s’écrit :
\begin{equation}
\mathcal{L}_{\text{CE}} = - \sum_{c=1}^{C} y_{c} \log p_{c},
\end{equation}
où $C$ est le nombre de classes, $y_{c}$ l’indicateur de classe (égal à 1 pour la classe correcte, 0 sinon) et $p_{c}$ la probabilité prédite pour la classe $c$.

La métrique principale utilisée pour comparer les modèles est le \textbf{taux de bonne classification}, défini comme la proportion de prédictions correctes sur un ensemble donné :
\begin{equation}
\text{Taux de bonne classification} = \frac{N_{\text{correct}}}{N_{\text{total}}},
\end{equation}
où $N_{\text{correct}}$ est le nombre de signaux correctement classés et $N_{\text{total}}$ le nombre total de signaux évalués.

\section{Optimisation et calendriers d'apprentissage}
L’optimisation des paramètres des réseaux CNN/CRNN est réalisée à l’aide de l’algorithme Adam. Pour un paramètre $w$ et un gradient $g_{t}$ à l’itération $t$, les mises à jour sont données par :
\begin{equation}
\begin{aligned}
m_{t} &= \beta_{1} m_{t-1} + (1-\beta_{1}) g_{t}, \\
v_{t} &= \beta_{2} v_{t-1} + (1-\beta_{2}) g_{t}^{2}, \\
\hat{m}_{t} &= \frac{m_{t}}{1-\beta_{1}^{t}}, \quad
\hat{v}_{t} = \frac{v_{t}}{1-\beta_{2}^{t}}, \\
w_{t+1} &= w_{t} - \eta \, \frac{\hat{m}_{t}}{\sqrt{\hat{v}_{t}} + \epsilon},
\end{aligned}
\end{equation}
où $\beta_{1}$ et $\beta_{2}$ sont les coefficients de moyennes mobiles, $\eta$ le taux d’apprentissage et $\epsilon$ un terme de stabilisation numérique.

Pour le modèle AudioMAE, un schéma plus avancé est adopté, conformément à la configuration détaillée dans \texttt{TRAINING\_SUMMARY\_REPORT.md} :
\begin{itemize}
  \item optimiseur AdamW (taux d’apprentissage initial $1\times 10^{-4}$, \emph{weight decay} 0{,}05) ;
  \item plan de \emph{learning rate} de type \textbf{cosine annealing} avec redémarrages, et phase de \emph{warm-up} en début d’entraînement ;
  \item entraînement sur 100~époques avec taille de lot 16 sur GPU.
\end{itemize}

Le schéma d’optimisation est couplé à des techniques de régularisation fortes (Mixup, CutMix, \emph{label smoothing}, dropout), ce qui explique le comportement de généralisation particulièrement favorable observé pour AudioMAE (validation mieux performante que l’entraînement).

\section{Hyperparamètres clés}
\begin{table}[H]
  \centering
  \begin{tabular}{@{}lcc@{}}
    \toprule
    Hyperparamètre & Valeur & Commentaire \\
    \midrule
    Taux d'apprentissage initial & $1\times10^{-3}$ & Optimiseur Adam \\
    Taille de lot & 32 & Nombre d'exemples par itération \\
    Nombre d'époques & 150 & Entraînement pour CNN et CRNN \\
    Facteur de \emph{dropout} & $0{,}5$ & Régularisation sur les couches denses \\
    \bottomrule
  \end{tabular}
  \caption{Hyperparamètres d'entraînement utilisés pour les modèles CNN/CRNN.}
  \label{tab:hyperparametres}
\end{table}

Pour AudioMAE, les hyperparamètres clés sont les suivants (configuration SereneSense) :
\begin{itemize}
  \item \textbf{Entrée} : spectrogrammes de Mel $128\times128$ calculés sur des segments de 10~secondes (16~kHz, FFT 1024, pas de 160~échantillons) ;
  \item \textbf{Architecture} : encodeur ViT 12 couches (768 dimensions, 12 têtes), décodeur 8 couches (512 dimensions, 16 têtes), taille de patch $16\times16$ ;
  \item \textbf{Entraînement} : 100~époques, taille de lot 16, AdamW, scheduler cosinus avec redémarrages ;
  \item \textbf{Régularisation} : Mixup ($\alpha = 0{,}8$), CutMix ($\alpha = 1{,}0$), \emph{label smoothing} 0{,}1, dropout 0{,}5, augmentations de type SpecAugment sur les spectrogrammes.
\end{itemize}

\section{Procédure d'entraînement}
\begin{algorithm}[H]
  \DontPrintSemicolon
  \SetAlgoLined
  \KwIn{Tenseurs MFCC $X$, étiquettes $Y$, nombre d'époques $T$}
  \For{$t \gets 1$ \KwTo $T$}{
    Mélanger les mini-batchs\;
    \For{chaque batch $(x_{b}, y_{b})$}{
      Appliquer les augmentations et normaliser\;
      Propagation avant, calcul des logits\;
      Calculer la perte $\mathcal{L}_{\text{CE}}$\;
      Rétropropagation, mise à jour des paramètres avec Adam\;
    }
    Évaluer sur validation toutes les $k$ époques\;
  }
  \caption{Procédure d'entraînement générique.}
  \label{alg:entrainement}
\end{algorithm}

\chapter{Expérimentations et résultats}\label{chap:results}

\section{Protocoles expérimentaux}
Les expériences sont conduites sur GPU, à partir des scripts et configurations du dépôt SereneSense. Deux \textbf{campagnes principales} ont été réalisées :
\begin{itemize}
  \item une première campagne dédiée aux \textbf{baselines MFCC} (réseau convolutionnel et réseau convolutionnel récurrent), entraînés pendant 150~époques sur le split d’entraînement MAD et évalués régulièrement sur l’ensemble de validation (configuration historique) ;
  \item une seconde campagne dédiée au \textbf{modèle AudioMAE}, entraîné pendant 100~époques sur la configuration finale de MAD (5\,464 échantillons d’entraînement, 965 de validation), avec un protocole moderne d’optimisation (AdamW, scheduler cosinus, régularisation forte).
\end{itemize}

Pour toutes les configurations, les modèles sont finalement évalués sur le jeu de test (1\,037 échantillons) à l’aide de plusieurs métriques : précision, rappel, F1 et taux de bonne classification. Les courbes complètes (pertes, exactitudes, comparaison de modèles) sont sauvegardées dans \texttt{docs/reports/} et les historiques d’entraînement dans le répertoire \texttt{outputs/history/}.

\section{Définitions des métriques}
Pour une classe donnée, avec vrai positifs $TP$, faux positifs $FP$, faux négatifs $FN$ :
\begin{align}
\text{Précision} &= \frac{TP}{TP + FP}, &
\text{Rappel} &= \frac{TP}{TP + FN}, \\
\text{F1} &= 2 \times \frac{\text{Précision} \times \text{Rappel}}{\text{Précision} + \text{Rappel}}, &
\text{Exactitude} &= \frac{\sum_{c} TP_{c}}{\sum_{c} (TP_{c}+FP_{c}+FN_{c})}.
\end{align}
Le \textbf{taux de bonne classification} correspond à l'exactitude globale et constitue l’indicateur principal utilisé pour comparer les performances du réseau convolutionnel et du réseau convolutionnel récurrent. L'analyse du \textbf{taux de confiance} fixe un seuil de probabilité (70\,\%) au-delà duquel une prédiction est acceptée ; l’impact sur précision et rappel est étudié.

\section{Performances globales}
\begin{table}[H]
  \centering
  \begin{tabular}{@{}lccc@{}}
    \toprule
    Modèle & Exactitude val. & Exactitude test & Écart val--train \\
    \midrule
    Réseau convolutionnel (CNN MFCC) & 66{,}88\,\% & -- & -1{,}57\,\% \\
    Réseau convolutionnel récurrent (CRNN MFCC) & 73{,}21\,\% & -- & -1{,}68\,\% \\
    Modèle AudioMAE (transformeur) & \textbf{82{,}15\,\%} & -- & \textbf{+12{,}38\,\%} \\
    \bottomrule
  \end{tabular}
  \caption{Comparaison des performances sur MAD (la colonne test est à compléter après exécution finale).}
  \label{tab:perf-modeles}
\end{table}

\placeholderfigure[fig:training-curves]{Courbes d'entraînement}{Courbes exactitude/perte pour les trois modèles (CNN, CRNN, AudioMAE) issues de SereneSense}
\placeholderfigure[fig:model-bar]{Comparaison de modèles}{Barres comparatives entre réseau convolutionnel, réseau convolutionnel récurrent et AudioMAE}
\placeholderfigure[fig:confusion]{Matrice de confusion 7$\times$7}{Matrice sur validation ou test}

\subsection{Analyse détaillée des résultats AudioMAE}
Le modèle AudioMAE atteint une exactitude de validation de \textbf{82{,}15~\%} après 100~époques d’entraînement, pour une exactitude d’entraînement de 69{,}77~\%. La \emph{generalization gap} (validation $>$ entraînement) est de \textbf{+12{,}38~\%}, ce qui est remarquable et indique une excellente capacité de généralisation :
\begin{itemize}
  \item la perte de validation (0{,}8693) est inférieure à la perte d’entraînement (0{,}9763), signe d’une absence de sur-apprentissage ;
  \item les techniques de régularisation (Mixup, CutMix, \emph{label smoothing}, dropout) jouent pleinement leur rôle en empêchant le modèle de mémoriser le bruit ;
  \item le modèle reste stable sur l’ensemble des 100~époques, sans divergence ni oscillations fortes des métriques.
\end{itemize}

Les rapports générés automatiquement (\texttt{TRAINING\_SUMMARY\_REPORT.md}, \texttt{FINAL\_RESULTS.md}) fournissent des tableaux récapitulatifs prêts à être intégrés dans le mémoire (fichiers CSV et figures au format PNG), ainsi que des analyses textuelles détaillées (qualité de la généralisation, impact des hyperparamètres, pistes d’amélioration).

\section{Analyse du seuil de confiance}
La probabilité minimale pour accepter une prédiction est fixée à 0{,}70. On définit :
\begin{equation}
\text{Taux de confiance} = \frac{\text{nombre de prédictions } p \geq 0{,}70}{\text{nombre total de prédictions}}.
\end{equation}
Une courbe précision–rappel conditionnée par le seuil est tracée ; le point d'exploitation retenu vise à conserver une précision élevée tout en maintenant un rappel satisfaisant. Les classes dominantes (véhicules militaires, hélicoptères) tendent à conserver un bon rappel, tandis que les classes plus difficiles (communications, bruit de fond) bénéficient de l’utilisation d’un seuil pour réduire les faux positifs.

\section{Analyse qualitative}
Les erreurs résiduelles se concentrent principalement sur les segments courts ou très bruités, ainsi que sur les transitions bruit de fond $\rightarrow$ véhicule distant. Les représentations temps–fréquence mal classées montrent des bandes fréquentielles partagées entre certaines classes proches, par exemple entre hélicoptères et avions de chasse. Une analyse qualitative des cartes de caractéristiques confirme que le réseau convolutionnel récurrent sépare mieux ces classes que le réseau purement convolutionnel, grâce à la prise en compte du contexte temporel.

\section{Performance temps réel et empreinte}
Du point de vue de la complexité :
\begin{itemize}
  \item les baselines CNN/CRNN restent de taille modérée (respectivement $\sim$242~k et $\sim$1{,}5~M de paramètres), ce qui les rend immédiatement compatibles avec des plateformes embarquées peu puissantes, au prix d’une précision limitée ;
  \item le modèle AudioMAE est nettement plus volumineux (111~M de paramètres, $\sim$424~Mo en FP32), mais il est accompagné d’un pipeline d’optimisation complet (export ONNX, quantification INT8, scripts de benchmarking) qui permet de réduire drastiquement son empreinte et sa latence sans sacrifier la précision.
\end{itemize}

Les mesures effectuées sur le poste de développement (CPU x86\_64, ONNX Runtime) montrent une \textbf{latence totale moyenne d’environ 46~ms} par fenêtre (prétraitement $\approx 22$~ms, inférence $\approx 24$~ms) pour le modèle AudioMAE en format ONNX FP32, soit une marge très confortable par rapport à l’objectif de 500~ms. Après quantification INT8, la taille du modèle est réduite d’un facteur proche de 4 (environ 83--106~Mo selon la version), et les projections sur Raspberry~Pi~5 indiquent des latences attendues de 260 à 340~ms par fenêtre de 10~s avec une consommation mémoire de l’ordre de 800~Mo. Les détails de ces mesures et projections sont présentés au chapitre~\ref{chap:deployment}.

\chapter{Déploiement sur systèmes embarqués}\label{chap:deployment}

\section{Stratégie générale}
Le déploiement cible une carte de type Raspberry~Pi~5 (processeur ARM Cortex-A76, 8\,Go de mémoire) équipée d’un micro-casque USB. L'objectif est de conserver un taux de bonne classification élevé tout en garantissant une latence inférieure à 500~ms par fenêtre audio et une empreinte mémoire compatible avec les ressources disponibles.

Au-delà d’une simple faisabilité théorique, le projet a abouti à la mise en place d’une \textbf{chaîne de déploiement complète}, décrite et implémentée dans le dépôt SereneSense :
\begin{itemize}
  \item scripts d’export du modèle AudioMAE depuis PyTorch vers ONNX (\texttt{export\_to\_onnx.py}) ;
  \item quantification dynamique INT8 du modèle ONNX (\texttt{quantize\_onnx.py}) ;
  \item validation fonctionnelle et de performance sur PC (\texttt{test\_deployment.py}, \texttt{PERFORMANCE\_REPORT.md}) ;
  \item scripts dédiés à la préparation et au déploiement sur Raspberry~Pi~5 (\texttt{rpi\_preprocessing.py}, \texttt{rpi\_deploy.py}, \texttt{rpi\_setup.sh}) ;
  \item documentation détaillée pour l’utilisateur final (\texttt{RPi5\_DEPLOYMENT\_GUIDE.md}, \texttt{QUICKSTART\_DEPLOYMENT.md}, \texttt{DEPLOYMENT\_SUMMARY.md}).
\end{itemize}

\section{Export et optimisation du format Open Neural Network Exchange}
Le modèle entraîné (dans la version finale, l’architecture AudioMAE) est exporté depuis PyTorch vers le format Open Neural Network Exchange (ONNX, opset~14) en conservant des axes dynamiques pour la taille de lot. La procédure mise en place comprend :
\begin{enumerate}
  \item le chargement du meilleur point de contrôle (\texttt{best\_model\_audiomae\_000.pth} ou \texttt{checkpoint\_audiomae\_099.pth}) ;
  \item la création d’un tenseur d’entrée factice de taille $(1, 1, 128, 128)$ représentant un spectrogramme de Mel normalisé ;
  \item l’export en ONNX avec vérification automatique de la cohérence des sorties par rapport au modèle PyTorch (écart numérique $\leq 10^{-6}$).
\end{enumerate}

Une quantification post-entraînement en entiers 8 bits (INT8) est ensuite appliquée pour réduire la taille du modèle et accélérer l’inférence, tout en contrôlant la perte de précision. Les scripts de quantification génèrent deux fichiers principaux :
\begin{itemize}
  \item un modèle ONNX FP32 (\texttt{audiomae\_fp32.onnx}) de l’ordre de 325--424~Mo ;
  \item un modèle quantifié INT8 (\texttt{audiomae\_int8.onnx}) d’environ 80--110~Mo, soit une réduction de taille proche de $4\times$.
\end{itemize}

\begin{table}[H]
  \centering
  \begin{tabular}{@{}lccc@{}}
    \toprule
    Format & Taille & Latence projetée & Commentaire \\
    \midrule
    ONNX en précision simple (FP32) & \textasciitilde325--424\,Mo & $\approx 46$~ms (PC) & Modèle exporté depuis PyTorch, référence de validation \\
    ONNX quantifié en entiers 8 bits & \textasciitilde80--110\,Mo & 260--340~ms (RPi5, estimée) & Modèle optimisé pour l’embarqué (INT8) \\
    \bottomrule
  \end{tabular}
  \caption{Effets de la quantification et projections de performance pour AudioMAE.}
  \label{tab:onnx-quant}
\end{table}

\placeholderfigure[fig:onnx-pipeline]{Pipeline ONNX}{Chaîne export $\rightarrow$ quantification $\rightarrow$ validation}

\section{Configuration Raspberry Pi 5}
Les étapes recommandées pour la mise en place du système sur Raspberry~Pi~5, telles que décrites dans les guides \texttt{RPi5\_DEPLOYMENT\_GUIDE.md} et \texttt{QUICKSTART\_DEPLOYMENT.md}, sont les suivantes :
\begin{enumerate}
  \item Installer Raspberry~Pi~OS 64 bits (Bookworm+), activer SSH et configurer un micro USB 16~kHz.
  \item Installer les dépendances : \texttt{onnxruntime} (version ARM optimisée), \texttt{numpy}, \texttt{librosa}, \texttt{soundfile}, \texttt{scipy}, \texttt{pyaudio}, ainsi que les bibliothèques système audio nécessaires.
  \item Transférer le modèle quantifié (\texttt{audiomae\_int8.onnx}) et les scripts d'inférence temps réel (\texttt{rpi\_preprocessing.py}, \texttt{rpi\_deploy.py}, \texttt{batch\_test.py}) dans un répertoire dédié sur la carte.
  \item Exécuter le script d’installation \texttt{rpi\_setup.sh} qui automatise la création de l’environnement virtuel, l’installation des bibliothèques Python et les tests de vérification.
  \item Mesurer la latence sur un nombre suffisant d’itérations et vérifier qu’elle reste compatible avec la contrainte temps réel (\(<500\)~ms par fenêtre de 10~secondes), tout en surveillant la consommation mémoire et la température.
\end{enumerate}

\placeholderfigure[fig:rpi-setup]{Installation RPi5}{Schéma du montage matériel et micro}

\section{Pipeline temps réel}
Le pipeline embarqué appliqué sur Raspberry~Pi~5 reprend fidèlement les étapes du pipeline d’entraînement, adaptées au contexte temps réel. Dans la version finale, il utilise les \textbf{spectrogrammes de Mel} associés à AudioMAE ; une version alternative basée sur les MFCC (pour les baselines CNN/CRNN) reste possible mais n’est pas détaillée ici. Le pipeline comprend :
\begin{enumerate}
  \item \textbf{Capture audio} : acquisition de segments de 10~secondes, en mono, 16~kHz, à partir d’un microphone USB, via \texttt{PyAudio}.
  \item \textbf{Prétraitement} : calcul du spectrogramme de Mel (128~bandes, FFT 1024, pas de 160~échantillons), passage en échelle logarithmique puis normalisation (centrage, réduction).
  \item \textbf{Inférence ONNX Runtime} : redimensionnement du spectrogramme en tenseur $(1,1,128,128)$, passage dans le modèle \texttt{audiomae\_int8.onnx} via le fournisseur \texttt{CPUExecutionProvider}, obtention des logits puis des probabilités par \emph{softmax}.
  \item \textbf{Décision} : application d’un seuil de confiance, typiquement 0{,}70, sur la probabilité maximale ; en dessous du seuil, la prédiction est rejetée ou marquée comme incertaine, au-dessus une alerte est générée et journalisée.
\end{enumerate}

\begin{algorithm}[H]
  \DontPrintSemicolon
  \SetAlgoLined
  \KwIn{Flux audio continu}
  \While{système actif}{
    Acquérir un segment audio de 10~s (16~kHz, mono)\;
    Calculer le spectrogramme de Mel $(128\times128)$ et le normaliser\;
    Passer le tenseur $(1,1,128,128)$ dans le modèle ONNX (INT8)\;
    Calculer les probabilités par \emph{softmax} et la classe la plus probable\;
    Appliquer le seuil $p \geq 0{,}70$ ; publier la classe ou rejeter\;
    Journaliser les latences et probabilités\;
  }
  \caption{Boucle d'inférence embarquée.}
  \label{alg:inference}
\end{algorithm}

\section{Validation embarquée et tests terrain}
Un protocole de tests terrain inclut, par exemple : (i) des scénarios de bruit urbain, (ii) des passages de véhicules à différentes vitesses et distances, (iii) des sons faibles en arrière-plan. Dans le cadre de ce projet, une première validation a été menée sur poste de développement (\texttt{PERFORMANCE\_REPORT.md}, \texttt{QUICK\_PERFORMANCE\_SUMMARY.md}) afin d’anticiper le comportement sur Raspberry~Pi~5 :
\begin{itemize}
  \item \textbf{Latence totale} mesurée à 46{,}01~ms en moyenne (prétraitement + inférence) sur PC, soit un facteur 10{,}9 plus rapide que la cible de 500~ms ;
  \item \textbf{Mémoire} utilisée d’environ 588~Mo pour le modèle FP32, ce qui laisse une marge confortable sur une carte 8~Go ;
  \item \textbf{Projections Raspberry~Pi~5} : latence totale estimée entre 260 et 340~ms avec le modèle INT8, mémoire autour de 800~Mo, précision attendue légèrement inférieure à 82{,}15~\% mais supérieure à 80~\%.
\end{itemize}

Ces résultats indiquent que le système est \textbf{compatible avec un usage temps réel embarqué} et qu’il dispose d’une marge importante pour intégrer des modules complémentaires (journalisation avancée, alertes réseau, interface de supervision). Les tests terrain sur rover NOMAD, avec de véritables enregistrements de véhicules, constituent l’étape suivante nécessaire pour valider le comportement du système dans des conditions opérationnelles réelles.

\chapter{Discussion et analyse critique}\label{chap:discussion}

\section{Bilan par rapport aux objectifs}
Les objectifs formulés en introduction portaient sur la mise en place d’un prototype de classification sonore pour le rover NOMAD, l’évaluation d’architectures d’apprentissage profond adaptées à l’audio (réseaux convolutionnels, réseaux convolutionnels récurrents et, dans la phase la plus récente, un modèle de type transformeur AudioMAE) et l’étude de la faisabilité d’un déploiement embarqué sur Raspberry~Pi~5.

Sur ces différents volets, le projet a permis d’atteindre plusieurs jalons importants :
\begin{itemize}
  \item \textbf{Baselines MFCC (CNN/CRNN).} Les architectures historiques basées sur des tenseurs MFCC (avec coefficients delta et delta-delta) ont été ré-implémentées proprement dans le dépôt SereneSense (\texttt{src/core/models/legacy}) et entraînées de manière systématique. Les résultats de validation obtenus (66{,}88~\% pour le CNN, 73{,}21~\% pour le CRNN) confirment l’intérêt d’une modélisation temporelle explicite via les couches récurrentes.
  \item \textbf{Modèle AudioMAE.} L’introduction du modèle transformeur AudioMAE, opérant sur des spectrogrammes de Mel de taille $128\times128$ sur 10~secondes d’audio, constitue la principale avancée en termes de performance, avec une exactitude de validation de 82{,}15~\% et une généralisation remarquable (validation meilleure que l’entraînement).
  \item \textbf{Chaîne logicielle et reproductibilité.} L’ensemble du pipeline (préparation des données, entraînement, évaluation, génération de rapports, export ONNX, quantification, scripts de déploiement) a été industrialisé dans un projet Python structuré, accompagné de tests, de configurations YAML et de documentation.
  \item \textbf{Préparation au déploiement embarqué.} Un plan de déploiement détaillé, ainsi que des scripts prêts à l’emploi pour Raspberry~Pi~5, ont été réalisés et validés sur poste de développement, avec des projections de performance réalistes démontrant la faisabilité temps réel.
\end{itemize}

La chaîne de prétraitement basée sur les MFCC, delta et delta-delta a été déterminante pour les baselines CNN/CRNN, tandis que les spectrogrammes de Mel et les mécanismes d’attention d’AudioMAE se sont révélés plus adaptés à la capture de motifs complexes sur des segments audio longs propres aux scènes militaires.

\section{Comparaison à l'état de l'art}
Par rapport aux approches les plus récentes de la littérature (modèles pré-entraînés de type YamNet ou architectures de type transformeur pour l’audio~\cite{gong2021ast,huang2022audiomae,chen2022beats}), le projet se situe dans une démarche progressive :
\begin{itemize}
  \item dans un premier temps, les baselines CNN/CRNN, bien que plus simples, ont permis d’obtenir des performances solides sur MAD (66{,}88~\% et 73{,}21~\%), en ligne avec ce que l’on peut attendre de réseaux supervisés entraînés sur un dataset de taille moyenne ;
  \item dans un second temps, l’adoption d’AudioMAE rapproche le système de l’état de l’art en classification audio, avec une exactitude de 82{,}15~\% malgré l’absence de pré-entraînement massif sur des bases de type AudioSet.
\end{itemize}

Les résultats obtenus montrent que les transformeurs pour l’audio apportent un gain significatif (+8{,}94~points par rapport au CRNN) tout en restant compatibles avec un déploiement embarqué après optimisation (quantification, ONNX Runtime). Ils constituent ainsi un compromis intéressant entre précision, robustesse et complexité pour des applications de surveillance acoustique militaire.

\section{Limites et risques}
\begin{itemize}
  \item \textbf{Variabilité acoustique} : bien que riche, la base MAD ne couvre pas toutes les conditions météorologiques, topographiques ou d’équipement microphonique rencontrées sur le terrain ; une collecte complémentaire sur rover NOMAD sera nécessaire pour couvrir davantage de cas réels.
  \item \textbf{Classes proches} : une confusion résiduelle subsiste entre certaines classes spectrales proches (par exemple hélicoptères et avions de chasse, véhicules militaires et camions), en particulier pour des sons éloignés ou noyés dans le bruit de fond.
  \item \textbf{Événements rares} : des événements impulsifs rares (tirs, explosions) peuvent être sous-représentés dans la base de données, ce qui limite la capacité du modèle à généraliser sur ces cas sans sur-apprentissage ou données supplémentaires.
  \item \textbf{Pré-entraînement absent} : le modèle AudioMAE a été entraîné \emph{from scratch} sur MAD, ce qui plafonne vraisemblablement la performance autour de 82--85~\%; un pré-entraînement auto-supervisé sur de grandes bases publiques pourrait encore améliorer les résultats.
  \item \textbf{Déploiement partiellement évalué} : la faisabilité embarquée a été étudiée en détail et validée par des mesures sur poste de développement, mais des mesures complètes sur carte Raspberry~Pi~5 et en conditions de terrain restent à réaliser pour finaliser la validation.
\end{itemize}

\section{Considérations éthiques et opérationnelles}
L'usage sur des scénarios de surveillance militaire impose : (i) transparence sur les limites du modèle, (ii) calibration fine du seuil pour éviter des décisions erronées, (iii) conformité aux règles d'engagement et à la protection des civils, (iv) journalisation et auditabilité des prédictions.

\section{Perspectives techniques}
\begin{itemize}
  \item \textbf{Enrichissement de la base de données} : collecter des enregistrements supplémentaires dans des conditions variées (météo, types de microphones, distances, scénarios réels sur NOMAD) afin d’améliorer la robustesse du modèle et de mieux couvrir les différentes situations opérationnelles.
  \item \textbf{Amélioration des modèles} : explorer des variantes plus profondes de réseaux convolutionnels ou des architectures hybrides plus légères, ainsi que des techniques de pondération des classes ou de pertes focales pour traiter les déséquilibres et mieux gérer les classes rares.
  \item \textbf{Pré-entraînement et transfert} : étudier l’apport de modèles pré-entraînés de type transformeur pour l’audio (AudioMAE pré-entraîné, BEATs, AST), en les comparant systématiquement aux baselines CNN/CRNN et au modèle AudioMAE entraîné uniquement sur MAD.
  \item \textbf{Détection continue} : mettre en place une détection sur fenêtres plus courtes (par exemple 2 à 3~secondes) avec agrégation des décisions sur une fenêtre glissante, afin de permettre une réaction plus précoce aux événements sonores et une meilleure localisation temporelle.
  \item \textbf{Intégration système} : intégrer le module de détection sonore avec les autres capteurs du rover NOMAD (vision, GPS, IMU) dans une logique de fusion de capteurs, afin de fournir une perception multimodale plus fiable.
\end{itemize}

\chapter{Conclusion et perspectives}\label{chap:conclusion}

Ce mémoire a présenté la conception et l’implémentation d'un système de classification sonore destiné au rover NOMAD, en s’appuyant sur une base de données spécialisée pour les applications militaires (MAD) et sur des techniques d’apprentissage profond allant des réseaux convolutionnels classiques aux architectures transformeur de dernière génération. Le premier chapitre a fusionné l’introduction générale, la présentation de l’organisme d’accueil et l’état de l’art, afin de situer le projet à la fois dans son contexte industriel (Avionav, rover NOMAD) et scientifique (classification audio, modèles profonds, transformeurs).

Les chapitres suivants ont détaillé la préparation du jeu de données et le prétraitement des signaux audio (MFCC, coefficients delta et delta-delta, spectrogrammes de Mel), puis la conception de plusieurs modèles de référence : un réseau de neurones convolutionnel, un réseau de neurones convolutionnel récurrent et, dans un second temps, un modèle AudioMAE basé sur une architecture de type Vision Transformer appliquée à l’audio. Les résultats expérimentaux ont montré une progression nette des performances :
\begin{itemize}
  \item les baselines CNN et CRNN atteignent respectivement 66{,}88~\% et 73{,}21~\% d’exactitude sur l’ensemble de validation MAD, confirmant que la modélisation temporelle explicite améliore la discrimination entre classes sonores militaires ;
  \item le modèle AudioMAE atteint 82{,}15~\% d’exactitude de validation, avec une généralisation particulièrement favorable (validation meilleure que l’entraînement), ce qui situe le système à un niveau de performance compatible avec un usage opérationnel.
\end{itemize}

Au-delà des résultats de classification, une contribution importante de ce travail réside dans la \textbf{structuration du code et du pipeline expérimental} : préparation automatisée de MAD, configuration des expériences via fichiers YAML, scripts d’entraînement et d’évaluation, génération de rapports LaTeX/Markdown, visualisation des courbes et stockage des historiques d’entraînement. Cette industrialisation, incarnée dans le dépôt SereneSense, garantit la reproductibilité des expériences et prépare le terrain à de futurs travaux de recherche et d’ingénierie.

Sur le plan du déploiement, un plan détaillé pour Raspberry~Pi~5 a été élaboré et outillé : export du modèle AudioMAE vers ONNX, quantification INT8, scripts de déploiement et de test, guides d’installation et de dépannage. Les mesures réalisées sur poste de développement indiquent une latence totale d’environ 46~ms pour le modèle ONNX FP32, et les projections sur Raspberry~Pi~5 (environ 260--340~ms avec le modèle INT8) respectent confortablement la contrainte de 500~ms par fenêtre de 10~secondes. Ces éléments montrent que la mise en œuvre embarquée est non seulement réaliste, mais également robuste en termes de marge de performance.

Les perspectives de ce travail incluent l’enrichissement de la base de données MAD par de nouveaux scénarios (conditions météorologiques variées, configurations microphoniques différentes, enregistrements sur rover NOMAD), l’amélioration des architectures existantes (pondération des classes, pertes focales, modèles hybrides plus légers) et l’exploration de modèles pré-entraînés plus avancés pour l’audio (AudioMAE pré-entraîné, BEATs, AST). À plus long terme, l’intégration du module de perception acoustique avec d’autres capteurs (vision, GPS, IMU) du rover ouvrira la voie à une perception multimodale robuste pour la surveillance de zones sensibles.

En réunissant rigueur méthodologique, structuration logicielle professionnelle et orientation pratique vers la robotique mobile, ce mémoire offre un socle solide pour le développement et l’amélioration continue d’un module de perception acoustique embarqué au service du rover NOMAD et, plus largement, des systèmes autonomes de surveillance acoustique en milieu militaire.


% ========== Annexes ==========
\appendix
\chapter{Liste des abréviations}

\begin{longtable}{@{}ll@{}}
\toprule
Abréviation & Signification \\
\midrule
IA & Intelligence Artificielle \\
CNN & Convolutional Neural Network \\
CRNN & Convolutional Recurrent Neural Network \\
UGV & Unmanned Ground Vehicle \\
MFCC & Mel-Frequency Cepstral Coefficients \\
ZRC & Zero Crossing Rate \\
SVM & Support Vector Machine \\
KNN & K-Nearest Neighbors \\
RF & Random Forest \\
RNN & Recurrent Neural Networks \\
LSTM & Long Short-Term Memory \\
GRU & Gated Recurrent Unit \\
YamNet & Yet Another Mobile Network for audio \\
Audio-MAE & Masked Autoencoder for Audio \\
AST & Audio Spectrogram Transformer \\
MAD & Military Audio Detection (dataset) \\
\bottomrule
\end{longtable}

\chapter{Checkpoints et artefacts}\label{app:checkpoints}

\begin{itemize}
  \item \texttt{outputs/phase1/cnn\_baseline.pth} : checkpoint du réseau de neurones convolutionnel (baseline CNN).
  \item \texttt{outputs/phase1/crnn\_baseline.pth} : checkpoint du réseau de neurones convolutionnel récurrent (baseline CRNN).
  \item \texttt{outputs/training\_config.json} : fichier de configuration de l’entraînement.
  \item \texttt{outputs/history/} : historiques d’entraînement (courbes de perte et de taux de bonne classification).
  \item \texttt{outputs/plots/} : figures et visualisations utilisées dans le chapitre des résultats.
\end{itemize}

\chapter{Scripts et reproduction}\label{app:scripts}

\section{Entraînement}
\begin{lstlisting}[language=bash,caption={Entraînement du réseau convolutionnel (CNN) sur MAD}]
python scripts/train_legacy_model.py \
  --model cnn \
  --dataset mad \
  --data-dir data/processed/mad \
  --epochs 150 \
  --batch-size 32
\end{lstlisting}

\begin{lstlisting}[language=bash,caption={Entraînement du réseau convolutionnel récurrent (CRNN) sur MAD}]
python scripts/train_legacy_model.py \
  --model crnn \
  --dataset mad \
  --data-dir data/processed/mad \
  --epochs 150 \
  --batch-size 16
\end{lstlisting}

\section{Évaluation}
\begin{lstlisting}[language=bash,caption={Évaluation d'un modèle CNN ou CRNN sur MAD}]
python scripts/evaluate_legacy_model.py \
  --model-path outputs/phase1/cnn_baseline.pth \
  --dataset mad \
  --data-dir data/processed/mad
\end{lstlisting}


% ========== Bibliographie ==========
\nocite{*}
\printbibliography

\end{document}
