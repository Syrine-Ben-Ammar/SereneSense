\chapter{Conclusion et perspectives}\label{chap:conclusion}

Ce mémoire a présenté la conception et l’implémentation d'un système de classification sonore destiné au rover NOMAD, en s’appuyant sur une base de données spécialisée pour les applications militaires (MAD) et sur des techniques d’apprentissage profond allant des réseaux convolutionnels classiques aux architectures transformeur de dernière génération. Le premier chapitre a fusionné l’introduction générale, la présentation de l’organisme d’accueil et l’état de l’art, afin de situer le projet à la fois dans son contexte industriel (Avionav, rover NOMAD) et scientifique (classification audio, modèles profonds, transformeurs).

Les chapitres suivants ont détaillé la préparation du jeu de données et le prétraitement des signaux audio (MFCC, coefficients delta et delta-delta, spectrogrammes de Mel), puis la conception de plusieurs modèles de référence : un réseau de neurones convolutionnel, un réseau de neurones convolutionnel récurrent et, dans un second temps, un modèle AudioMAE basé sur une architecture de type Vision Transformer appliquée à l’audio. Les résultats expérimentaux ont montré une progression nette des performances :
\begin{itemize}
  \item les baselines CNN et CRNN atteignent respectivement 66{,}88~\% et 73{,}21~\% d’exactitude sur l’ensemble de validation MAD, confirmant que la modélisation temporelle explicite améliore la discrimination entre classes sonores militaires ;
  \item le modèle AudioMAE atteint 82{,}15~\% d’exactitude de validation, avec une généralisation particulièrement favorable (validation meilleure que l’entraînement), ce qui situe le système à un niveau de performance compatible avec un usage opérationnel.
\end{itemize}

Au-delà des résultats de classification, une contribution importante de ce travail réside dans la \textbf{structuration du code et du pipeline expérimental} : préparation automatisée de MAD, configuration des expériences via fichiers YAML, scripts d’entraînement et d’évaluation, génération de rapports LaTeX/Markdown, visualisation des courbes et stockage des historiques d’entraînement. Cette industrialisation, incarnée dans le dépôt SereneSense, garantit la reproductibilité des expériences et prépare le terrain à de futurs travaux de recherche et d’ingénierie.

Sur le plan du déploiement, un plan détaillé pour Raspberry~Pi~5 a été élaboré et outillé : export du modèle AudioMAE vers ONNX, quantification INT8, scripts de déploiement et de test, guides d’installation et de dépannage. Les mesures réalisées sur poste de développement indiquent une latence totale d’environ 46~ms pour le modèle ONNX FP32, et les projections sur Raspberry~Pi~5 (environ 260--340~ms avec le modèle INT8) respectent confortablement la contrainte de 500~ms par fenêtre de 10~secondes. Ces éléments montrent que la mise en œuvre embarquée est non seulement réaliste, mais également robuste en termes de marge de performance.

Les perspectives de ce travail incluent l’enrichissement de la base de données MAD par de nouveaux scénarios (conditions météorologiques variées, configurations microphoniques différentes, enregistrements sur rover NOMAD), l’amélioration des architectures existantes (pondération des classes, pertes focales, modèles hybrides plus légers) et l’exploration de modèles pré-entraînés plus avancés pour l’audio (AudioMAE pré-entraîné, BEATs, AST). À plus long terme, l’intégration du module de perception acoustique avec d’autres capteurs (vision, GPS, IMU) du rover ouvrira la voie à une perception multimodale robuste pour la surveillance de zones sensibles.

En réunissant rigueur méthodologique, structuration logicielle professionnelle et orientation pratique vers la robotique mobile, ce mémoire offre un socle solide pour le développement et l’amélioration continue d’un module de perception acoustique embarqué au service du rover NOMAD et, plus largement, des systèmes autonomes de surveillance acoustique en milieu militaire.
